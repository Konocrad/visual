\documentclass{article}

\usepackage[english,russian]{babel}

\title{Code Style}
\author{author: Daniyar N. Zhetygeldenov, IA-032;\\ email: reedix@bk.ru}

\begin{document}

\maketitle

\section{Вступление}
Данный отчет содержит рекомендации написания кода, то есть "Code Style". 
Целью этих правил является приятная читаемость кода. 
\section{C++}
Currently, the C++14 version is used \cite{CPP}.
\subsection{Пример программы}

\begin{tabular}{|p {400}|}
\hline
\\
\begin{verbatim}
#include <iostream>
 
using namespace std;

void printArr(int *arr, int size);

int main() 
{
    setlocale(LC_ALL, "rus");
 
    int arr[10]; // объявили массив на 10 ячеек
 
    cout << "Введите 10 чисел для заполнения массива: " << endl;
 
    for (int i = 0; i < 10; i++) 
    {
        cin >> arr[i]; // "читаем" элементы в массив
    }
 
    for (int i = 0; i < 10; i++) 
    {
        for (int j = 0; j < 9; j++) 
        {
            if (arr[j] > arr[j + 1]) 
            {
                int b = arr[j]; // создали дополнительную переменную
                arr[j] = arr[j + 1]; // меняем местами
                arr[j + 1] = b; // значения элементов
            }
        }
    }
 
    
    printArr(arr, 10);

    return 0;
}

void printArr(int *arr, int size)
{
    cout << "Массив в отсортированном виде: ";
    for (int i = 0; i < 10; i++) 
    {
        cout << arr[i] << " "; // выводим элементы массива
    }
}
\end{verbatim}
\\
\hline      
\end{tabular}

\subsection{Имена функций}
Имена функций обязательно должны показывать смысл, предназначение или действие которое выполняет эта функция.\\
\begin{tabular}{|c|c}
\hline
    \\
    void printArr(int *arr, int size)\\
    \\
\hline      
\end{tabular}

\subsection{Переменным-итераторам следует давать имена i, j, k и т. д.}

\begin{tabular}{|c|c}
\hline
    \\
    for (int i = 0; i < 10); i++)\\
    \\
\hline      
\end{tabular}

\subsection{Фигурные скобки}
Фигурные скобки должны стоять в новой строке и являться единственным символом в строке.\\
\begin{tabular}{|p {320}|}
\hline
\\
    \begin{verbatim}

    for (int i = 0; i < 10; i++) 
    {
        cout << arr[i] << " "; // выводим элементы массива
    }
            
    \end{verbatim}
\\
\hline      
\end{tabular}

\subsection{Вывод на экран}
Вывод на экран производится через "cout".
Его содержимое должно быть понятно пользователю.\\
\\
\begin{tabular}{|p {320}|}
\hline
\begin{verbatim}

cout << arr[i] << " "; // выводим элементы массива

\end{verbatim}
\\
\hline      
\end{tabular}
\\
\\
\\
\\
\subsection{Пробелы}
Пробелы ставятся между операторами и операндами.\\
\\
\begin{tabular}{|p {320}|}
\hline
\begin{verbatim}

for (int i = 0; i < 10; i++)

\end{verbatim}
\\
\hline      
\end{tabular}

\subsection{Return}
"Return" ставится в конце функции main и возвращает "0".\\
\\
\begin{tabular}{|p {320}|}
\hline
\begin{verbatim}

return 0;

\end{verbatim}
\\
\hline      
\end{tabular}
\\
\\
В случае ошибки программа завершается с кодом 1, за исключением использования флагов и т.д..\\
\\
\begin{tabular}{|p {320}|}
\hline
\begin{verbatim}

return 1;

\end{verbatim}
\\
\hline      
\end{tabular}

\subsection{Имена переменных}
Имена переменных должны иметь смысл понятный для других людей, но не должны быть громоздкими.
Краткое и информативное название - то, что нужно для имени переменной. \\
\\
\begin{tabular}{|p {320}|}
\hline
\begin{verbatim}

int value;
char name;
int index;
int distanceAB;
int distanceCD;

\end{verbatim}
\\
\hline      
\end{tabular}
\section{Вывод}
В ходе данной лабораторной работы я научился создавать и редактировать текстовый файл через cистему компьютерной вёрстки LATEX.  
\begin{thebibliography}{1}
\bibitem{CPP}
ISO/IEC 14882 Programming languages - C++.\\
\end{thebibliography}

\end{document}